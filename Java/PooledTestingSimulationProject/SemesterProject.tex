%%%%%%%%%%%%%%%%%%%%%%%%%%%%%%%%%%%%%%%%%
%
% CMPT 435
% Lab 5
%
%%%%%%%%%%%%%%%%%%%%%%%%%%%%%%%%%%%%%%%%%

%%%%%%%%%%%%%%%%%%%%%%%%%%%%%%%%%%%%%%%%%
% Short Sectioned Assignment
% LaTeX Template
% Version 1.0 (5/5/12)
%
% This template has been downloaded from: http://www.LaTeXTemplates.com
% Original author: % Frits Wenneker (http://www.howtotex.com)
% License: CC BY-NC-SA 3.0 (http://creativecommons.org/licenses/by-nc-sa/3.0/)
% Modified by Alan G. Labouseur  - alan@labouseur.com
%
%%%%%%%%%%%%%%%%%%%%%%%%%%%%%%%%%%%%%%%%%

%----------------------------------------------------------------------------------------
%	PACKAGES AND OTHER DOCUMENT CONFIGURATIONS
%----------------------------------------------------------------------------------------

\documentclass[letterpaper, 10pt]{article} 

\usepackage[english]{babel} % English language/hyphenation
\usepackage{listings}
\usepackage{graphicx}
\usepackage[lined,linesnumbered,commentsnumbered]{algorithm2e}
\usepackage{listings}
\usepackage{fancyhdr} % Custom headers and footers
\pagestyle{fancyplain} % Makes all pages in the document conform to the custom headers and footers
\usepackage{lastpage}
\usepackage{url}


\fancyhead{} % No page header - if you want one, create it in the same way as the footers below
\fancyfoot[L]{} % Empty left footer
\fancyfoot[C]{page \thepage\ of \pageref{LastPage}} % Page numbering for center footer
\fancyfoot[R]{}

\renewcommand{\headrulewidth}{0pt} % Remove header underlines
\renewcommand{\footrulewidth}{0pt} % Remove footer underlines
\setlength{\headheight}{13.6pt} % Customize the height of the header

%----------------------------------------------------------------------------------------
%	TITLE SECTION
%----------------------------------------------------------------------------------------

\newcommand{\horrule}[1]{\rule{\linewidth}{#1}} % Create horizontal rule command with 1 argument of height

\title{	
   \normalfont\normalsize 
   \textsc{CMPT 435 - Fall 2021 - Dr. Labouseur} \\[10pt] % Header stuff.
   \horrule{0.5pt} \\[0.25cm] 	% Top horizontal rule
   \huge Semester Project \\   % Assignment title
   \horrule{0.5pt} \\[0.25cm] 	% Bottom horizontal rule
}

\author{Jake Vissicchio \\ \normalsize Jake.Vissicchio1@Marist.edu}

\date{\normalsize\today} 	% Today's date.
\usepackage{multirow}
\begin{document}

\maketitle % Print the title

%----------------------------------------------------------------------------------------
%   CONTENT SECTION
%----------------------------------------------------------------------------------------

% - -- -  - -- -  - -- -  -

\section{Results}
\noindent
Case 1: Represents a pool where there were no infected samples. \\
Case 2: Represents when there is one or more infected samples within one half of the pool.\\
Case 3: Represents where there are infected samples within both halves of the pool.\\
\\
Expected Results: \\
Case (1): 125 × 0.8500 = 106.25 instances requiring 107 tests \\
Case (2): 125 × 0.1496 = 18.70 instances requiring 131 tests \\
Case (3): 125 × 0.0004 = 0.05 round up to 1 instance requiring 11 tests \\
\\
\noindent
The table below represents the simulated results using different total\\ population sizes...\\

\begin{tabular}{ |p{2cm}||p{2cm}|p{2cm}|p{2cm}|p{2cm} |}
 \hline
 \multicolumn{5}{|c|}{Pooled Testing Results Using Different Population Sizes} \\
 \hline
 Population: & Total Tests: & Case 1 Total: & Case 2 Total: & Case 3 Total: \\
 \hline
1,000   &  243    &  106 & 18 & 1\\
10,000 & 2,410 & 1,064 & 175 & 11   \\
100,000 & 23,960 & 10,642 & 1,780 & 78 \\
1M & 240,028 & 106,326 & 17,928 & 746 \\
 \hline
\end{tabular}
\\
As shown, the results are close to the expected results. Also, each time we\\
add a 0 to the population we often see each case amount go up by a new digit\\
which fits.\\

\section{Binomial Distribution \& \\Hypergeometric Distribution}
\noindent
\subsection{Binomial Distribution}
Definition:\\
Binomial Distribution is a discrete probability distribution of the possible\\ number of successful outcomes in a given number of trials in each of which \\
there is the same probability of success.\\
\\
Formula: \\
\[P_{x} = {n \choose x} p^{x} q^{n-x}\]
$P$	=	binomial probability\\
$x$	=	number of times for a specific outcome within n trials\\
${n \choose x}$	=	number of combinations\\
$p$	=	probability of success on a single trial\\
$q$	=	probability of failure on a single trial\\
$n$	=	number of trials\\

\subsection{Hypergeometric Distribution}
Definition: \\
Hypergeometric Distribution is a discrete probability distribution that describes \\
the probability of success given a number of trials, without replacement.\\\
\\
Formula: \\
\[P_{x} = \frac{ _{s_{p}}C_{s_{s}}* _{n_{p}-s_{p}}C_{n_{s}-s_{s}}}{_{n_{p}}C_{n_{s}}}\]
$P$	=	Hypergeometric probability\\
$s_{p}$	= number of successes in the population\\
$s_{s}$	= number of successes in the sample\\
$n_{p}$	= size of the population\\
$n_{s}$	= size of the sample\\

\subsection{Difference}
For binomial distribution, the probability is the same for every trial\\
because it is WITH replacement. For Hypergeometric Distribution, the \\ probability changes each trial because it is WITHOUT replacement. This will\\
cause each trial after one another to be different.\\
The difference between the probabilities are very slim so it is usually\\
okay to ignore for most applications.\\
It will be best to use Hypergeometric Distribution when the population is small\\ since that is when we will see a difference due to its dependent nature.\\

\section{How Can I Improve My Simulation?}
\subsection{Adding False Positives/Negatives}
Perhaps the adding of false positives and negatives will add another layer\\
of realism within my simulation. While false positives are not too much too\\
worry about besides taking up more tests, false negatives can be extremely\\
dangerous since it allows an infected individual to do things as if they were \\
not infected.

\subsection{Vicinity Testing}
Perhaps the adding of vicinity testing will also add realism within my\\ simulation. By vicinity testing, I mean having the pools be grouped together \\
by people that live close to one another. This will add many more Case 3s to the
results as those that are infected are likely to infect those around them.\\

\subsection{Missing Testee}
Perhaps adding a checker to see if an individual missed their test can prove\\
to be beneficial for my simulation. At Marist, there were many times when \\
students would skip their testing in favor of doing their Semester-Projects \\
and would have to be reminded to go as soon as possible after or else there \\
be punishment. This can be quite bad because they can be positive without \\
knowing if they skipped their testing.\\


\end{document}
