%%%%%%%%%%%%%%%%%%%%%%%%%%%%%%%%%%%%%%%%%
%
% CMPT 435
% Lab 2
%
%%%%%%%%%%%%%%%%%%%%%%%%%%%%%%%%%%%%%%%%%

%%%%%%%%%%%%%%%%%%%%%%%%%%%%%%%%%%%%%%%%%
% Short Sectioned Assignment
% LaTeX Template
% Version 1.0 (5/5/12)
%
% This template has been downloaded from: http://www.LaTeXTemplates.com
% Original author: % Frits Wenneker (http://www.howtotex.com)
% License: CC BY-NC-SA 3.0 (http://creativecommons.org/licenses/by-nc-sa/3.0/)
% Modified by Alan G. Labouseur  - alan@labouseur.com
%
%%%%%%%%%%%%%%%%%%%%%%%%%%%%%%%%%%%%%%%%%

%----------------------------------------------------------------------------------------
%	PACKAGES AND OTHER DOCUMENT CONFIGURATIONS
%----------------------------------------------------------------------------------------

\documentclass[letterpaper, 10pt]{article} 

\usepackage[english]{babel} % English language/hyphenation
\usepackage{listings}
\usepackage{graphicx}
\usepackage[lined,linesnumbered,commentsnumbered]{algorithm2e}
\usepackage{listings}
\usepackage{fancyhdr} % Custom headers and footers
\pagestyle{fancyplain} % Makes all pages in the document conform to the custom headers and footers
\usepackage{lastpage}
\usepackage{url}


\fancyhead{} % No page header - if you want one, create it in the same way as the footers below
\fancyfoot[L]{} % Empty left footer
\fancyfoot[C]{page \thepage\ of \pageref{LastPage}} % Page numbering for center footer
\fancyfoot[R]{}

\renewcommand{\headrulewidth}{0pt} % Remove header underlines
\renewcommand{\footrulewidth}{0pt} % Remove footer underlines
\setlength{\headheight}{13.6pt} % Customize the height of the header

%----------------------------------------------------------------------------------------
%	TITLE SECTION
%----------------------------------------------------------------------------------------

\newcommand{\horrule}[1]{\rule{\linewidth}{#1}} % Create horizontal rule command with 1 argument of height

\title{	
   \normalfont\normalsize 
   \textsc{CMPT 435 - Fall 2021 - Dr. Labouseur} \\[10pt] % Header stuff.
   \horrule{0.5pt} \\[0.25cm] 	% Top horizontal rule
   \huge Assignment Two \\   % Assignment title
   \horrule{0.5pt} \\[0.25cm] 	% Bottom horizontal rule
}

\author{Jake Vissicchio \\ \normalsize Jake.Vissicchio1@Marist.edu}

\date{\normalsize\today} 	% Today's date.
\usepackage{multirow}
\begin{document}

\maketitle % Print the title

%----------------------------------------------------------------------------------------
%   CONTENT SECTION
%----------------------------------------------------------------------------------------

% - -- -  - -- -  - -- -  -

\section{Results}

\noindent



\begin{tabular}{ |p{4cm}||p{4cm}|p{4cm}|p{4cm}|  }
 \hline
 \multicolumn{3}{|c|}{Sorting Magic Items} \\
 \hline
 Sort Type& Number of Comparisons Recorded & Asymtotic Running Time\\
 \hline
 Selection Sort   & 225415    & $O(n^2)$\\
 Insertion Sort&   114927 & $O(n^2)$   \\
 Merge Sort & 5443 & $O(n*log(n))$\\
 Quick Sort    & 2975 & $O(n*log(n))$\\
 \hline
\end{tabular}
\\

\section{Explanations}
\noindent
\\
Selection Sort is $O(n^2)$ because in order to sort it relies on the use\\
of a nested for loop that we will have to go through in order to find position\\ of the smallest value (or in this case coming first in the alphabet.\\
\\
Insertion Sort is $O(n^2)$ because in order to sort it relies on the use\\
of a nested loop. It has the potential to be faster than Selection Sort if\\
the array is mostly sorted since it can skip more swaps and comparisons\\
\\
Merge Sort is $O(n*log(n)$ because it makes use of divide and conquer.\\
The array of n elements is repeatedly halved while sorting which results\\ 
in $O(log(n))$ and then the array of n elements is merged together resulting\\
in $O(n)$. Putting these two together we get $O(n*log(n))$.\\
\\
Quick Sort is $O(n*log(n)$ because it also makes use of divide and conquer\\
but differently from Merge Sort. While Merge Sort recursively divides until\\ array reaches a length of one then conquers. Quick Sort recursively divides \\
and conquers by partitioning the halves around a pivot value. By making use \\
of a good partition value the partitioning will have a complexity of $O(n)$.\\
With Sith Lord Partitioning such as picking the first or last value as the\\
pivot value every time it will increase the complexity even further.\\
As before, the array of n elements being halved will result in $O(log(n))$.\\
Putting these two together we get $O(n*log(n))$.\\

\end{document}

